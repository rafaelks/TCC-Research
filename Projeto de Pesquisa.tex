%% abtex2-modelo-projeto-pesquisa.tex, v-1.7.1 laurocesar
%% Copyright 2012-2013 by abnTeX2 group at http://abntex2.googlecode.com/ 
%%
%% This work may be distributed and/or modified under the
%% conditions of the LaTeX Project Public License, either version 1.3
%% of this license or (at your option) any later version.
%% The latest version of this license is in
%%   http://www.latex-project.org/lppl.txt
%% and version 1.3 or later is part of all distributions of LaTeX
%% version 2005/12/01 or later.
%%
%% This work has the LPPL maintenance status `maintained'.
%% 
%% The Current Maintainer of this work is the abnTeX2 team, led
%% by Lauro César Araujo. Further information are available on 
%% http://abntex2.googlecode.com/
%%
%% This work consists of the files abntex2-modelo-projeto-pesquisa.tex
%% and abntex2-modelo-references.bib
%%

% ------------------------------------------------------------------------
% ------------------------------------------------------------------------
% abnTeX2: Modelo de Projeto de pesquisa em conformidade com 
% ABNT NBR 15287:2011 Informação e documentação - Projeto de pesquisa -
% Apresentação 
% ------------------------------------------------------------------------ 
% ------------------------------------------------------------------------

\documentclass[
	12pt,
	oneside,
	a4paper,
	english,
	brazil,
]{abntex2}

% ---
% PACOTES
% ---

% ---
% Pacotes fundamentais 
% ---
\usepackage{cmap}				% Mapear caracteres especiais no PDF
\usepackage{lmodern}			% Usa a fonte Latin Modern
\usepackage[T1]{fontenc}		% Selecao de codigos de fonte.
\usepackage[utf8]{inputenc}		% Codificacao do documento (conversão automática dos acentos)
\usepackage{indentfirst}		% Indenta o primeiro parágrafo de cada seção.
\usepackage{color}				% Controle das cores
\usepackage{graphicx}			% Inclusão de gráficos
\usepackage{xcolor}
\usepackage{tabularx}
% ---

% ---
% Pacotes de citações
% ---
\usepackage[brazilian,hyperpageref]{backref}	 % Paginas com as citações na bibl
\usepackage[alf]{abntex2cite}	% Citações padrão ABNT

% --- 
% CONFIGURAÇÕES DE PACOTES
% --- 

% ---
% Diretório das imagens
% ---
\graphicspath{ {Images/} }

% ---
% Configurações do pacote backref
% Usado sem a opção hyperpageref de backref
\renewcommand{\backrefpagesname}{Citado na(s) página(s):~}
% Texto padrão antes do número das páginas
\renewcommand{\backref}{}
% Define os textos da citação
\renewcommand*{\backrefalt}[4]{
	\ifcase #1 %
		Nenhuma citação no texto.%
	\or
		Citado na página #2.%
	\else
		Citado #1 vezes nas páginas #2.%
	\fi}%
% ---

% ---
% Informações de dados para CAPA e FOLHA DE ROSTO
% ---
\titulo{Aplicação que utiliza Beacons para automatizar a chamada em sala de aula}
\autor{Rafael Kellermann Streit}
\orientador{Eurico Antunes}
\local{Brasil}
\data{2016}
\instituicao{%
  FACCAT - Faculdades Integradas de Taquara
  \par
  Graduação em Sistemas de Informação}
\tipotrabalho{Projeto de Pesquisa}
% O preambulo deve conter o tipo do trabalho, o objetivo, 
% o nome da instituição e a área de concentração 
\preambulo{Aplicação que faz a marcação de presença dos alunos em sala de aula de forma automatizada, utilizando Beacons.}
% ---

% ---
% Configurações de aparência do PDF final

% informações do PDF
\makeatletter
\hypersetup{
     	%pagebackref=true,
		pdftitle={\@title}, 
		pdfauthor={\@author},
    	pdfsubject={\imprimirpreambulo},
	    pdfcreator={LaTeX with abnTeX2},
		pdfkeywords={abnt}{latex}{abntex}{abntex2}{projeto de pesquisa}{beacon}{bluetooth}{ios}{mobile}, 
		colorlinks=false,       		% false: boxed links; true: colored links
    	linkcolor=black,          	% color of internal links
    	citecolor=black,        		% color of links to bibliography
    	filecolor=magenta,      		% color of file links
		urlcolor=blue,
		bookmarksdepth=4
}
\makeatother
% --- 

% --- 
% Espaçamentos entre linhas e parágrafos 
% --- 

% O tamanho do parágrafo é dado por:
\setlength{\parindent}{1.3cm}

% Controle do espaçamento entre um parágrafo e outro:
\setlength{\parskip}{0.2cm}  % tente também \onelineskip

% ---
% compila o indice
% ---
\makeindex
% ---

% ----
% Início do documento
% ----
\begin{document}

% Retira espaço extra obsoleto entre as frases.
\frenchspacing 

% ----------------------------------------------------------
% ELEMENTOS PRÉ-TEXTUAIS
% ----------------------------------------------------------
% \pretextual

% ---
% Capa
% ---
\imprimircapa
% ---

% ---
% Folha de rosto
% ---
\imprimirfolhaderosto
% ---

% ---
% NOTA DA ABNT NBR 15287:2011, p. 4:
%  ``Se exigido pela entidade, apresentar os dados curriculares do autor em
%     folha ou página distinta após a folha de rosto.''
% ---

% ---
% inserir o sumario
% ---
\pdfbookmark[0]{\contentsname}{toc}
\tableofcontents*
% ---


% ----------------------------------------------------------
% ELEMENTOS TEXTUAIS
% ----------------------------------------------------------
\newpage
\textual

% ----------------------------------------------------------
% Introdução
% ----------------------------------------------------------
\chapter*[Introdução]{Introdução}
\addcontentsline{toc}{chapter}{Introdução}

Este trabalho de pesquisa e desenvolvimento consiste na criação de uma aplicação capaz de substituir a marcação de presença atual em salas de aula, com o objetivo de ter maior precisão, rapidez e facilidade. Para isto, é necessário utilizar uma tecnologia para a identificação da presença do aluno que, chama-se \emph{Beacon} e está presente em praticamente todos os \emph{smartphones} atuais com suporte a \emph{Bluetooth 4.0}.

Além da marcação de presença dos alunos, esta aplicação identifica pontos de interesse na instituição de ensino, informando aos gestores da instituição e demais interessados setores quais os locais mais frequentados pelos alunos e seus respectivos horários. Informações essas que são adquiridas também através de \emph{Beacons}.

% ----------------------------------------------------------
% Problematização  
% ----------------------------------------------------------

\chapter{Problematização}

Atualmente para que um professor registre a presença dos alunos em sua aula é necessário algum tipo de interação entre ele e o aluno. O método mais utilizado é a chamada em voz alta, em que o professor chama pelo nome um por um de seus alunos, e o aluno precisa responder, também em voz alta, assim que ouvir o seu nome. Ao retornar ao professor a comunicação a que foi chamado, o aluno sinaliza que está presente, em sala de aula. O professor faz o registro dessa presença manualmente, em um documento específico para esse registro. Outro método utilizado é o aluno assinar uma ata em todas as aulas, marcando sua presença. Ambos os métodos tem diversos problemas, como:

\begin{itemize}
    \item Tempo de aula significativo gasto para a marcação de presença;
    \item Recurso passível de fraude pois os alunos podem responder por outro colega na chamada, mesmo que este não esteja presente. O professor estando atento a chamada dos alunos verbal e ao registro das presenças, pode não se atentar a quem respondeu e dar a presença por engano a um aluno;
    \item Pode acontecer uma perda do documento em que é realizado o registro das presenças e os dados daquele dia, ou dos anteriores, serem perdidos, extraviados por diversos motivos como furto, desastres naturais, acidentes, entre outros);
    \item Em função do preenchimento ser manual, pode haver erro na marcação. O que dificulta o processo, por ser passível de equívocos, fraudes ou desatenções.
\end{itemize}

Além de todos estes fatores, ainda existe uma enorme vantagem em informatizar o processo: permitir que sejam feitas análises de forma ágil e rápida nas informações de chamada, algo inviável de ser fazer quando o processo é manual. Com as informações computadorizadas, o aluno poderia facilmente receber um alerta quando está faltando demais às aulas e a instituição poderia, por exemplo, verificar qual o dia de maior falta e atuar com maior agilidade. Outros aspectos também poderiam ser otimizados com o uso desta tecnologia como oferecimento de praticidade ao professor, além de segurança e transparência nos registros realizados,  


% ----------------------------------------------------------
% Estado da arte  
% ----------------------------------------------------------

\chapter{Estado da arte}

\section{Instituição educacional}

Nesta pesquisa, será considerado que uma instituição de ensino é qualquer instituição que tem como objetivo a educação de pessoas, com a presença de um tutor, professor ou instrutor. A pesquisa será realizada com instituições de ensino superior, entrentato o software pode atender também instituições de outros níveis de ensino, desde educação infantil até ensino fundamental e médio.

\subsection{Processo de chamada em sala de aula}

O processo de chamada em sala de aula pouco mudou na maioria das escolas nas últimas décadas. A pessoa responsável pela turma, seja ela um professor, um tutor ou um instrutor, tem um documento em que consta o nome de todos os alunos e a listagem dos dias que ocorrerão as aulas daquela matéria. Então a cada aula o professor deve realizar a chamada, momento em que chama nome por nome de aluno, registrando quando houver a falta de alguém, na data correspondente. Reconhece-se que este processo é bastante democrático e funciona em basicamente todos os casos, porém, existem algumas dificuldades neste método, como: a perda de tempo de aula para fazer a chamada, a falha humana que pode ser causada e muitos outros.

\section{Smartphones}

Em 2007, após o lançamento do iPhone, a popularização do termo \emph{smartphone} se tornou muito forte, mas o conceito é mais antigo do que isto \cite{smartphone-history-pictures}. Desde então, o país com maior número deles \cite{smartphone-numbers}.

Por mais que existam milhares de modelos diferentes no mundo (e cada dia esse número cresce mais), o Bluetooth é algo que pode ser incorporado em qualquer um, por ser uma tecnologia barata de se implantar, simples e prática, amplamente usada no mundo \cite{smartphone-android-models}.

A popularização de recursos baseados em \emph{Bluetooth}, tais como fones de ouvido, sensores cardíacos, caixas de som e integração com automóveis fazem com que o uso dessa tecnologia se torne igualmente requisitada.

Além disto, o número de smartphones no Brasil não para de crescer, ultrapassando a 168 milhões em 2016 \cite{smartphone-brazil-numbers}, representando mais de 80\% da população brasileira \cite{brazil-population-numbers}.

\section{Beacons}

A tecnologia dos \emph{Beacons} começou a ficar mais popular por volta de 2013 \cite{beacon-what-is-it-forbes}, quando a Apple lançou ela junto ao novo iPhone, o modelo 4S. Junto com o lançamento da tecnologia no iPhone, a Apple colocou \emph{Beacons} em mais de 250 de suas lojas físicas \cite{beacon-apple-store-case}, trazendo assim uma experiência diferenciada para os clientes que entravam na loja, olhavam os produtos e faziam pedidos.

Muitas pessoas se confundem que o \emph{Beacon} pode ser considerado bastante similar a tecnologia \emph{RFID (Radio-Frequency Identification)}, porém seu uso se difere em alguns aspectos importantes:

\begin{itemize}
    \item O \emph{Beacon} utiliza \emph{Bluetooth} para se comunicar, tornando-se assim uma tecnologia mais acessível, devido ao grande volume de dispositivos que já vem com a tecnologia. Estima-se que até 2018, cerca de 90\% dos dispositivos terão \emph{Bluetooth LTE (Low Energy)} \cite{beacon-devices-estimate-2018}.
    \item Os \emph{Beacons} são mais robustos (em questões de hardware e firmware), eles se tornariam mais caros do que \emph{RFID} para a utilização em uma linha de produção industrial, por exemplo, onde a quantidade de \emph{tags} utilizadas é imensa (uma por peça/produto). No sub-capítulo "Funcionamento" é explicado mais sobre a complexidade do dispositivo e porque isto seria mais difícil.
    \item O \emph{Beacon} fornece maior privacidade e opções de segurança ao usuário, pois ele requer alguns passos e autorizações para seu funcionamento, \cite{beacon-apple-store-case}, o que não é o caso de uma antena \emph{RFID} que lê qualquer tag que passa pelas ondas do leitor.
\end{itemize}

\subsection{Componentes de um beacon}

Existem diversos modelos diferentes de \emph{Beacons}. Além das funcionalidades básicas, que serão explicadas a seguir, eles podem conter funcionalidades adicionais, como por exemplo: \emph{GPS}, sensor de temperatura e conexão \emph{Wi-Fi} \cite{beacon-sensors-easy-kontakt}.

Mas para o funcionamento básico de um \emph{Beacon}, é necessário que ele tenha:

\begin{itemize}
    \item Bateria: o Beacon precisa de energia para seu funcionamento, pois ficará emitindo sinais de tempo em tempo (isto é configurável) para que os dispositivos recebam suas informações.
    \item Pequeno computador: para o funcionamento entre bateria, \emph{Bluetooth} e \emph{firmware}.
    \item \emph{Bluetooth}: tecnologia utilizada para comunicação do \emph{Beacon} com os demais dispositivos.
    \item \emph{Firmware}: \emph{software} que irá configurar o \emph{Beacon}, tanto suas informações (que estão descritas abaixo) quanto sua potência de sinal e também seu intervalo de disparo.
\end{itemize}

Além disto, é necessário que o \emph{Beacon} tenha algumas informações para seu mínimo funcionamento:

\begin{itemize}
    \item \emph{UDID}: identificador do \emph{Beacon} com tipo \emph{hash}. Pode ser igual ao de outros \emph{Beacons}, desde que a aplicação tenha esse objetivo.
    \item \emph{Major}: identificador numérico do \emph{Beacon} para uso na aplicação.
    \item \emph{Minor}: identificador numérico do \emph{Beacon} para uso na aplicação.
\end{itemize}

Estas informações contidas no \emph{Beacon}, são utilizadas para sua identificação dentro da aplicação, pois com ela é possível saber em qual contexto o usuário está inserido. No \emph{software} desenvolvido no presente projeto, todos os \emph{Beacons} podem ter o mesmo identificador, mas com \emph{major} diferentes, pois assim a aplicação precisa ficar apenas escutando um identificador mas com a diferença no valor do \emph{major} será possível saber em qual sala de aula ou local da instituição o aluno está.

\begin{figure}[h]
\centering
\includegraphics[width=0.7\textwidth]{estimote-beacon}
    \caption{
        \emph{Beacon} da \emph{Estimote} com os componentes: capa de borracha (impermeável), \emph{hardware} com \emph{firmware} e \emph{Bluetooth}, bateria e protetor no fundo
    }
\end{figure}

\subsection{Funcionamento}

O funcionamento do \emph{Beacon}, depende principalmente de três elementos que são configuráveis: o intervalo de disparo de sinal (normalmente em torno de 600ms) e a força do sinal, que é o que irá determinar a distância máxima atingida pelo \emph{Beacon}. O sistema operacional \emph{iOS} faz o \emph{scan} dos dispositivos a cada 1 segundo, isto significa que se o o \emph{Beacon} estiver com intervalo de disparo em 330ms, o iOS irá receber três vezes o sinal naquele momento. Esta configuração normalmente é feita para lugares mais barulhentos, onde o sinal pode não chegar sempre ao celular, por motivos de outras ondas interferirem na comunicação \cite{beacon-how-it-works-estimote}.

\begin{figure}[h]
\centering
\includegraphics[width=0.85\textwidth]{beacon-how-it-works}
    \caption{
        A imagem mostra o comportamento do \emph{Beacon} com três telefones, um escutando-o, outro com \emph{Bluetooth} desligado e outro escutando outro \emph{Beacon}. Fonte: Autor.
    }
\end{figure}

\subsection{Segurança}

O \emph{Beacon} é considerada uma tecnologia muito segura, principalmente pelo fato de eladepender da aprovação do usuário para seu funcionamento, pois em primeiro lugar, é necessário que o usuário tenha instalado um aplicativo que consiga ficar escutando as ondas de certos \emph{Beacons} e mesmo que o usuário tenha o aplicativo instalado, ele ainda precisa aprovar o monitoramento dos \emph{Beacons}, podendo aprovar e restringir a qualquer momento \cite{beacon-what-is-it-forbes}.

\section{Tecnologias}

\subsection{Swift}

\emph{Swift} é uma linguagem \emph{open-source} extremamente moderna que foi liberada para o público em 2014, mas seu desenvolvimento inicou em 2010. Ela foi desenvolvida pela Apple, mas não é utilizada exclusivamente para seu ecossistema \cite{swift-about}. Conquistou rapidamente os desenvolvedores de aplicativos para iOS e hoje é utilizada na maioria dos novos aplicativos.

\emph{Swift} é uma linguagem estática, compilada e com uma sintaxe muito moderna que traz bastante segurança e rapidez aos desenvolvedores \cite{swift-about}.

\subsection{Web Services}

\emph{Web service} é uma interface que descreve uma coleção de operações que são acessíveis através da internet padronizados em \emph{XML}, \emph{JSON} ou algum tipo de padrão de informações conhecido, desde que todas as operações sigam o mesmo padrão. Ele providencia todos os meios para a comunicação da aplicação com o servidor, tornando a comunicação padrão e centralizada para todas as aplicações do serviço \cite{introduction-web-services}.

Para o desenvolvimento dos \emph{web services}, serão utilizadas diversas tecnologias em conjunto, proporcionando assim um sistema facilmente escalável e seguro. Ele utilizará o padrão \emph{JSON} em seus serviços para que as mesmas possam ser utilizadas por todos os softwares que conversarão com a plataforma, seja ele um aplicativo ou um servidor, fornecendo ou pedindo informações.

\subsubsection{Infraestrutura}

Na infraestrutura, optou-se pela utilização de software livre e de código aberto, já conhecidos no mercado há anos, por serem mais seguros e estáveis.

Como servidor \emph{HTTP}, será utilizado o \emph{Nginx}, criado em 2002 com o objetivo de servir sites com um enorme volume de acessos. É um software bastante leve e utilizado mundialmente hoje em aplicações de grande escala \cite{nginx-book}.

O sistema operacional utilizado no servidor será o \emph{Ubuntu Server LTS 16.04}, também é um software livre e de código aberto. Também foi escolhido este sistema operacional pelo fato dos autores já terem conhecimento prévio na utilização da tecnologias nesta plataforma, facilitando o desenvolvimento.

\subsubsection{Django}

Na interface de aplicação do \emph{web service}, será utilizado \emph{Django}, que é um \emph{framework} utilizado em aplicações \emph{Python}, que fornece inúmeras facilidades como: \emph{ORM (Object-relational mapping)}, construção de \emph{API}, migrações de bancos de dados e um painel administrativo que se constrói automaticamente baseado no modelo do banco de dados.

\emph{Django} utiliza o padrão \emph{MTV (Model-Template-View)} e o princípio \emph{DRY (Don't repeat yourself)} para o desenvolvimento. Isto faz com que exista uma separação muito clara entre o modelo, o template e o design do projeto, tornando-se mais fácil de reutilizar código e testar unitariamente os métodos e classes \cite{django-wikipedia}. 

\subsubsection{JSON}

Toda a comunicação entre a \emph{API} e as aplicações será feita no padrão \emph{JSON (JavaScript Object Notation)}, deixando assim uma comunicação uniforme e utilizando um padrão amplamente conhecido no mundo inteiro, de fácil leitura e compreensão \cite{footnote-json}.



% ----------------------------------------------------------
% Justificativa
% ----------------------------------------------------------

\chapter{Justificativa}

Conforme o gráfico abaixo, pode-se notar que o crescimento do uso de \emph{smartphones} no mundo cresce a cada ano, evidenciando que grande parte da população tem, pelo menos, um \emph{smartphone}.

\begin{figure}[h]
\centering
\includegraphics[width=1.0\textwidth]{smartphone_sales}
    \caption{
        Número de vendas de smartphones no mundo e uma projeção para o futuro, até 2021 \cite{smartphone-numbers-image}.
    }
\end{figure}

Com estas informações pode-se entender que a maioria das pessoas que frequenta uma instituição de ensino, terá um \emph{smartphone} em mãos. Levando em consideração que a maioria dos \emph{smartphones} no mercado tem a tecnologia \emph{Bluetooth}, pode-se entender então que é uma tecnologia bastante presente nos dias atuais, tendendo a ser mais presente ainda com a entrada de novos dispositivos que aderirem a tecnologia \cite{bluetooth-devices}.

Neste cenário diante dos fatores citados no capítulo de problematização, foi concluído que pode-se obter soluções utilizando uma tecnologia como o \emph{Bluetooth}, otimizando o tempo gasto em sala de aula, deixando-a mais produtiva e dinâmica.

Além disto, é importante ressaltar que no modelo manual (em papel) de chamadas não existe uma maneira fácil de aplicar operações de \emph{Big Data}\footnote{Refere-se a um grande conjunto de dados armazenados. \cite{bigdata-wikipedia}} e \emph{BI}\footnote{O termo Business Intelligence (BI), inteligência de negócios, refere-se ao processo de coleta, organização, análise, compartilhamento e monitoramento de informações que oferecem suporte a gestão de negócios. \cite{bi-mean}} nas informações armazenadas. Com a informatização das informações, é possível que sejam levantados dados relevantes ainda não descobertas sobre as aulas e o comportamento dos alunos e professores na instituição de ensino.


% ----------------------------------------------------------
% Objetivos
% ----------------------------------------------------------

\chapter{Objetivos}

\section{Objetivos gerais}
Otimizar as atividades de registro de presenças de alunos (controle de frequência em sala de aula) através do desenvolvimento e disponibilização de um produto com a utilização de \emph{Beacons} e \emph{smartphones}.

\section{Objetivos específicos}
\begin{itemize}
  \item Desenvolver uma aplicação web para fornecimento de informações dos alunos aos aplicativos através de uma \emph{API}, além de fornecer as informações de presença aos professores em formato de relatório;
  \item Desenvolver uma aplicação para \emph{iOS} que será utilizada pelos alunos para a marcação de presença em sala de aula de forma automática, utilizando \emph{Beacons} que serão instalados em salas de aula e na instituição de ensino;
  \item Testar aplicação em uma sala de aula de alguma instituição de ensino com um grupo de alunos, durante pelo menos 3 meses;
\end{itemize}

% ----------------------------------------------------------
% Resultados Esperados
% ----------------------------------------------------------

\chapter{Resultados esperados}

Espera-se atingir o objetivo proposto, tornando viável um melhor controle de presença dos alunos em instituições de ensino, mais produtivo e eficaz do que o atual praticado pela maioria das instituições.
O sucesso desta aplicação bem como sua comercialização também é esperada, podendo assim ser aplicada em diversas instituições de ensino com cenários diferentes.


% ----------------------------------------------------------
% Metodologia
% ----------------------------------------------------------

\chapter{Metodologia}
\section{Proposta do sistema}

Este projeto propõe o desenvolvimento de uma aplicação para \emph{iOS}, que poderá ser publicada na \emph{App Store}\footnote{Loja oficinal de aplicativos para iOS, maiores informações em: https://developer.apple.com/app-store/}, onde alunos das instituições de ensino irão se autenticar para efetivamente usa-lo, para que a chamada em sala de aula seja feita de forma automática. Além desta aplicação, também será desenvolvido um servidor web, que poderá ser hospedado em qualquer serviço de \emph{Cloud Computing}\footnote{Computação em Nuvem, onde as preocupações com a largura de banda, espaço de armazenamento, poder de processamento, fiabilidade e segurança, são postas de parte\cite{cloud-computing-about}} (\emph{AWS}\footnote{https://aws.amazon.com/pt/}, por exemplo). Neste servidor estará toda a integração com o aplicativo para \emph{iOS} e demais plataformas futuramente, como \emph{Android}\footnote{Sistema Operacional Open Source desenvolvido pelo Google e disponível em: https://android.com/intl/pt\-BR\_br/}, por exemplo.

\subsection{Aplicativo iOS}

O aplicativo iOS é a aplicação na qual o aluno irá interagir, autenticando-se e vendo informações de suas disciplinas matriculadas, além de permitir acesso Bluetooth e de localização ao aplicativo, para que a presença do aluno seja marcada automaticamente nas disciplinas através dos \emph{Beacons} instalados em sala de aula.

O aluno após autenticar-se, poderá acompanhar quais disciplinas está matriculado em uma listagem e dentro de cada disciplina poderá ver informações importantes como: nome do professor, e-mail do professor, sala de aula, data da próxima aula, quantidade de presenças e faltas e se possível, as notas da disciplina.

\subsection{Servidor Web}

O servidor web servirá como um apoio fundamental para o funcionamento do aplicativo, pois nele estarão todos os dados sincronizados entre o \emph{software} e a instituição de ensino, além das informações providas do próprio aluno automaticamente, através dos \emph{Beacons}.

Além de armazenar os dados, ele terá uma \emph{API}\footnote{Application Programming Interface: interface padrão para comunicação entre os dispositivos \cite{footnote-api}.} que fornecerá as informações no formato \emph{JSON}\footnote{Padrão aberto de comunicação em formato de objeto JavaScript, de fácil leitura e compreensão \cite{footnote-json}.} para as aplicações citadas neste trabalho (e demais aplicações que queiram integrar). Toda a comunicação com o servidor é feita de forma segura, utilizando o protocolo \emph{SSL}\footnote{Secure Sockets Layer: protocolo seguro de comunicação, que criptografa as informações através de um certificado seguro e validado \cite{footnote-ssl}}. 

\section{Desenvolvimento do sistema}
\subsection{Análise}

Será feito uma análise dos elementos (e suas características) existentes em um ambiente de chamadas de uma instituição de ensino superior que podem ajudar na elaboração do projeto, com informações importantes e até mesmo obrigatórias para a elaboração da chamada automatizada, para não deixar de conter nenhuma informação relevante.

\subsection{Projeto}

Todo o processo de desenvolvimento e análise utilizará a metodologia ágil chamada \emph{Scrum}, método muito difundido hoje, visando ter mais flexibilidade nos processos, dividindo as tarefas em outras tarefas pequenas, permitindo assim ver o projeto como um todo, facilitando muito a projeção de prazos, funcionalidades e desempenho \cite{scrum}.

Com relação ao desenvolvimento a partir dos dados analisados, serão criados diagramas que demonstrarão a estrutura e fluxo da aplicação. Em seguida será feito o projeto da interface com \emph{mockups} do aplicativo para visualização.

\subsection{Codificação}

A codificação deste produto será utilizando o paradigma de orientação à objetos. As linguagens serão \emph{Swift} e \emph{Python}. Para que esta aplicação possa ser desenvolvida, será utilizado a \emph{IDE} \emph{Xcode} para criação do aplicativo \emph{iOS} e um editor de textos para a criação da aplicação web. Para realizar os testes e também utilizar a aplicação, será necessário pelo menos um dispositivo com \emph{iOS}, um navegador web e um \emph{Beacon}.

\subsection{Testes}

Tanto a aplicação web quanto a aplicação para \emph{iOS} terão testes unitários automatizados, além dos testes que serão realizados manualmente para a identificação da posição e potência de sinal do \emph{Beacon}, para encontrar assim o melhor cenário em uma sala de aula para seu funcionamento. Após o término do desenvolvimento inicial da aplicação, pretende-se testar com alguns alunos em uma sala de aula por pelo menos 3 meses, para assim certificar-se de que o sistema está realmente oferecendo o suporte necessário.

\subsection{Implantação}

O \emph{software} a ser desenvolvido será instalado em um servidor na nuvem \emph{Ubuntu Server LTS} e a aplicação \emph{iOS} inicialmente será distribuída através do \emph{TestFlight}, uma plataforma de distribuição fornecida para a \emph{Apple} enquanto a aplicação não está disponível para o público. Quando a aplicação estiver pronta para o público, ela será publicada na \emph{App Store}, sendo assim, acessível por todos.

\section{Manutenção do sistema}

Para a manutenção do software será utilizado o processo de desenvolvimento incremental, para o controle de versionamento, manutenção corretiva e evolutiva das versões lançadas (inclusive na fase de testes).

Para isto, será utilizado o software \emph{git}\footnote{Sistema de controle de versão amplamente utilizado no mundo, disponível em: https://git-scm.com} para controle de versão, na plataforma do \emph{GitHub} de forma privada. Nunca sendo alterado código diretamente em produção ou qualquer outro meio que não esteja vinculado ao controle de versão.

\section{Tecnologias e ferramentas}

O desenvolvimento do \emph{software} para \emph{iOS} será desenvolvimento na \emph{IDE} \footnote{Integrated Development Enviroment}, ou Ambiente Integrado para desenvolvimento) \emph{Xcode}, fornecido pela Apple para desenvolvimento de aplicativos \emph{iOS} e \emph{macOS}.

Na aplicação web, serão utilizadas ferramentas já muito maduras no mercado, com o objetivo de ter uma aplicação estável, de fácil manutenção e segura. A primeira camada de comunicação do servidor será feita com \emph{Nginx}\footnote{Servidor Web mundialmente conhecido e utilizado, disponível em: https://nginx.com}, seguida do \emph{Gunicorn}\footnote{Servidor Python WSGI HTTP para Linux, disponível em: http://gunicorn.org}, responsável por manter a aplicação feita em \emph{Django}\footnote{Framework Python amplamente utilizado, disponível em: https://djangoproject.com} sempre ligada, respondendo as demandas dos usuários. O \emph{Django} oferece soluções muito fáceis e eficientes para a construção de um banco de dados, que neste caso será o \emph{PostgreSQL}\footnote{Banco de Dados Open Source, disponível em: https://postgresql.org}, para a construção de um painel administrativo e também de um \emph{API}.

\section{Futuras versões e expansão}

No desenvolvimento inicial deste projeto, será feito um aplicativo para a plataforma \emph{iOS}, mas posteriormente poderão ser desenvolvidas versões para outros sistemas operacionais, como \emph{Android} e \emph{Windows Phone}, para uma maior utilização do \emph{software}.

Todos os \emph{web services} criados serão feitos de tal maneira que possam ser utilizadas os mesmos serviços para todas as plataformas, podendo mudar alguns parâmetros por plataforma.


% ----------------------------------------------------------
% Orçamento  
% ----------------------------------------------------------

\chapter{Orçamento}

\noindent
\begin{table}[ht]
    \caption{Investimento Inicial Fixo}
    
    \begin{tabularx}{\textwidth}{X|c|r|r}
    \hline
        \textbf{Descrição} &
        \textbf{Quantidade} & 
        \textbf{Valor unitário} & 
        \textbf{Valor total} \\
    \hline
        Apple MBPr 15" 16GB 500GB SSD&1&R\$18.000&R\$18.000 \\
        iPhone 7 128GB&2&R\$4.000&R\$8.000 \\
        Beacon Estimote&10&R\$120&R\$1.200 \\ [1ex]
    \hline
        \textbf{Total}&&&\textbf{R\$27.000} \\ [1ex]
    \end{tabularx}
\end{table}

\noindent
\begin{table}[ht]
    \caption{Investimento com recursos humanos}

    \begin{tabularx}{\textwidth}{X|l|c|r|r}
    \hline
        \textbf{Nome} & 
        \textbf{Função} & 
        \textbf{Quantidade} & 
        \textbf{Valor hora} &
        \textbf{Valor total} \\
    \hline
        Rafael K. Streit&Desenvolvedor &300&R\$100&R\$30.000\\
        Eurico Antunes&Orientador&60&R\$200&R\$12.000 \\
    \hline
        \textbf{Total} &&&& \textbf{R\$32.000} \\
    \end{tabularx}
\end{table}

\noindent
\begin{table}[ht]
    \caption{Investimento Mensal}
    \begin{tabularx}{\textwidth}{X|c|r}
    \hline
        \textbf{Descrição} &
        \textbf{Quantidade} &
        \textbf{Valor} \\
    \hline
        Servidor Ubuntu Server na AWS &1 &R\$100 \\
    \hline
        \textbf{Total}&&\textbf{R\$100} \\ [1ex]
    \end{tabularx}
\end{table}


% ----------------------------------------------------------
% Cronograma  
% ----------------------------------------------------------

\chapter{Cronograma}

\begin{figure}[h]
\centering
\includegraphics[width=1.0\textwidth]{cronograma}
\end{figure}


% ----------------------------------------------------------
% ELEMENTOS PÓS-TEXTUAIS
% ----------------------------------------------------------
\postextual

% ----------------------------------------------------------
% Referências bibliográficas
% ----------------------------------------------------------
\bibliography{Referencias}

% ----------------------------------------------------------
% Glossário
% ----------------------------------------------------------
%
% Consulte o manual da classe abntex2 para orientações sobre o glossário.
%
%\glossary

\end{document}
